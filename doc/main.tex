% generated by GAPDoc2LaTeX from XML source (Frank Luebeck)
\documentclass[a4paper,11pt]{report}

\usepackage{a4wide}
\sloppy
\pagestyle{myheadings}
\usepackage{amssymb}
\usepackage[latin1]{inputenc}
\usepackage{makeidx}
\makeindex
\usepackage{color}
\definecolor{FireBrick}{rgb}{0.5812,0.0074,0.0083}
\definecolor{RoyalBlue}{rgb}{0.0236,0.0894,0.6179}
\definecolor{RoyalGreen}{rgb}{0.0236,0.6179,0.0894}
\definecolor{RoyalRed}{rgb}{0.6179,0.0236,0.0894}
\definecolor{LightBlue}{rgb}{0.8544,0.9511,1.0000}
\definecolor{Black}{rgb}{0.0,0.0,0.0}

\definecolor{linkColor}{rgb}{0.0,0.0,0.554}
\definecolor{citeColor}{rgb}{0.0,0.0,0.554}
\definecolor{fileColor}{rgb}{0.0,0.0,0.554}
\definecolor{urlColor}{rgb}{0.0,0.0,0.554}
\definecolor{promptColor}{rgb}{0.0,0.0,0.589}
\definecolor{brkpromptColor}{rgb}{0.589,0.0,0.0}
\definecolor{gapinputColor}{rgb}{0.589,0.0,0.0}
\definecolor{gapoutputColor}{rgb}{0.0,0.0,0.0}

%%  for a long time these were red and blue by default,
%%  now black, but keep variables to overwrite
\definecolor{FuncColor}{rgb}{0.0,0.0,0.0}
%% strange name because of pdflatex bug:
\definecolor{Chapter }{rgb}{0.0,0.0,0.0}
\definecolor{DarkOlive}{rgb}{0.1047,0.2412,0.0064}


\usepackage{fancyvrb}

\usepackage{mathptmx,helvet}
\usepackage[T1]{fontenc}
\usepackage{textcomp}


\usepackage[
            pdftex=true,
            bookmarks=true,        
            a4paper=true,
            pdftitle={Written with GAPDoc},
            pdfcreator={LaTeX with hyperref package / GAPDoc},
            colorlinks=true,
            backref=page,
            breaklinks=true,
            linkcolor=linkColor,
            citecolor=citeColor,
            filecolor=fileColor,
            urlcolor=urlColor,
            pdfpagemode={UseNone}, 
           ]{hyperref}

\newcommand{\maintitlesize}{\fontsize{50}{55}\selectfont}

% write page numbers to a .pnr log file for online help
\newwrite\pagenrlog
\immediate\openout\pagenrlog =\jobname.pnr
\immediate\write\pagenrlog{PAGENRS := [}
\newcommand{\logpage}[1]{\protect\write\pagenrlog{#1, \thepage,}}
%% were never documented, give conflicts with some additional packages

\newcommand{\GAP}{\textsf{GAP}}

%% nicer description environments, allows long labels
\usepackage{enumitem}
\setdescription{style=nextline}

%% depth of toc
\setcounter{tocdepth}{1}





%% command for ColorPrompt style examples
\newcommand{\gapprompt}[1]{\color{promptColor}{\bfseries #1}}
\newcommand{\gapbrkprompt}[1]{\color{brkpromptColor}{\bfseries #1}}
\newcommand{\gapinput}[1]{\color{gapinputColor}{#1}}


\begin{document}

\logpage{[ 0, 0, 0 ]}
\begin{titlepage}
\mbox{}\vfill

\begin{center}{\maintitlesize \textbf{The \textsf{G2Comp} Package\mbox{}}}\\
\vfill

\hypersetup{pdftitle=The \textsf{G2Comp} Package}
\markright{\scriptsize \mbox{}\hfill The \textsf{G2Comp} Package \hfill\mbox{}}
{\Huge \textbf{Equivariant 2-complexes\mbox{}}}\\
\vfill

{\Huge Version 1.0.1\mbox{}}\\[1cm]
\mbox{}\\[2cm]
{\Large \textbf{Iv{\a'a}n Sadofschi Costa   \mbox{}}}\\
\hypersetup{pdfauthor=Iv{\a'a}n Sadofschi Costa   }
\end{center}\vfill

\mbox{}\\
{\mbox{}\\
\small \noindent \textbf{Iv{\a'a}n Sadofschi Costa   }  Email: \href{mailto://isadofschi@dm.uba.ar} {\texttt{isadofschi@dm.uba.ar}}\\
  Homepage: \href{http://mate.dm.uba.ar/~isadofschi} {\texttt{http://mate.dm.uba.ar/\texttt{\symbol{126}}isadofschi}}}\\
\end{titlepage}

\newpage\setcounter{page}{2}
{\small 
\section*{Copyright}
\logpage{[ 0, 0, 1 ]}
{\copyright} 2018 Iv{\a'a}n Sadofschi Costa. \mbox{}}\\[1cm]
\newpage

\def\contentsname{Contents\logpage{[ 0, 0, 2 ]}}

\tableofcontents
\newpage

 
\chapter{\textcolor{Chapter }{G2Comp}}\logpage{[ 1, 0, 0 ]}
\hyperdef{L}{X81497C9084D3B21B}{}
{
   
\section{\textcolor{Chapter }{Introduction}}\label{sec:introd}
\logpage{[ 1, 1, 0 ]}
\hyperdef{L}{X7DFB63A97E67C0A1}{}
{
  This package includes functions to construct, manipulate and study 2-complexes
with an admissible action of a finite group \mbox{\texttt{\mdseries\slshape G}}. }

  
\section{\textcolor{Chapter }{Components of a G 2-complex}}\logpage{[ 1, 2, 0 ]}
\hyperdef{L}{X7A4AAAF182B85997}{}
{
  

\subsection{\textcolor{Chapter }{GroupOfComplex}}
\logpage{[ 1, 2, 1 ]}\nobreak
\hyperdef{L}{X85A081AA84878F3A}{}
{\noindent\textcolor{FuncColor}{$\triangleright$\ \ \texttt{GroupOfComplex({\mdseries\slshape K})\index{GroupOfComplex@\texttt{GroupOfComplex}}
\label{GroupOfComplex}
}\hfill{\scriptsize (function)}}\\


 Returns the group acting on \mbox{\texttt{\mdseries\slshape K}}. }

 

\subsection{\textcolor{Chapter }{VerticesOfComplex}}
\logpage{[ 1, 2, 2 ]}\nobreak
\hyperdef{L}{X8341B434782A84CE}{}
{\noindent\textcolor{FuncColor}{$\triangleright$\ \ \texttt{VerticesOfComplex({\mdseries\slshape K})\index{VerticesOfComplex@\texttt{VerticesOfComplex}}
\label{VerticesOfComplex}
}\hfill{\scriptsize (function)}}\\


 Returns the set of vertices of \mbox{\texttt{\mdseries\slshape K}}. }

 

\subsection{\textcolor{Chapter }{EdgesOfComplex}}
\logpage{[ 1, 2, 3 ]}\nobreak
\hyperdef{L}{X7EC49AE37A3445BA}{}
{\noindent\textcolor{FuncColor}{$\triangleright$\ \ \texttt{EdgesOfComplex({\mdseries\slshape K})\index{EdgesOfComplex@\texttt{EdgesOfComplex}}
\label{EdgesOfComplex}
}\hfill{\scriptsize (function)}}\\


 Returns the set of edges of \mbox{\texttt{\mdseries\slshape K}}. }

 

\subsection{\textcolor{Chapter }{FacesOfComplex}}
\logpage{[ 1, 2, 4 ]}\nobreak
\hyperdef{L}{X78E0103884033568}{}
{\noindent\textcolor{FuncColor}{$\triangleright$\ \ \texttt{FacesOfComplex({\mdseries\slshape K})\index{FacesOfComplex@\texttt{FacesOfComplex}}
\label{FacesOfComplex}
}\hfill{\scriptsize (function)}}\\


 Returns the set of faces of \mbox{\texttt{\mdseries\slshape K}}. }

 

\subsection{\textcolor{Chapter }{LabelsOfComplex}}
\logpage{[ 1, 2, 5 ]}\nobreak
\hyperdef{L}{X865AEE5985B5251B}{}
{\noindent\textcolor{FuncColor}{$\triangleright$\ \ \texttt{LabelsOfComplex({\mdseries\slshape K})\index{LabelsOfComplex@\texttt{LabelsOfComplex}}
\label{LabelsOfComplex}
}\hfill{\scriptsize (function)}}\\


 Returns the set of labels of the orbits of \mbox{\texttt{\mdseries\slshape K}}. }

 }

  
\section{\textcolor{Chapter }{Representation in \textsf{GAP}}}\label{sec:representation}
\logpage{[ 1, 3, 0 ]}
\hyperdef{L}{X8615C5CF7FD92EB1}{}
{
  Most of the time it should not be necessary to have in mind how objects are
represented in \textsf{G2Comp}. An \emph{oriented edge} is a list \mbox{\texttt{\mdseries\slshape [e,s]}} where \mbox{\texttt{\mdseries\slshape e}} is an edge of \mbox{\texttt{\mdseries\slshape K}} and \mbox{\texttt{\mdseries\slshape s}} is $1$ or $-1$. An \emph{edge path} is represented as a list of oriented edges (the target of each edge has to be
equal to the source of the next edge). }

  
\section{\textcolor{Chapter }{Functions to construct G - 2-complexes}}\logpage{[ 1, 4, 0 ]}
\hyperdef{L}{X79ACC0DD7A41431C}{}
{
  In this section we describe the main functions of this package. 

\subsection{\textcolor{Chapter }{NewEquivariantTwoComplex}}
\logpage{[ 1, 4, 1 ]}\nobreak
\hyperdef{L}{X853470687AD20266}{}
{\noindent\textcolor{FuncColor}{$\triangleright$\ \ \texttt{NewEquivariantTwoComplex({\mdseries\slshape G})\index{NewEquivariantTwoComplex@\texttt{NewEquivariantTwoComplex}}
\label{NewEquivariantTwoComplex}
}\hfill{\scriptsize (function)}}\\


 Returns an empty \mbox{\texttt{\mdseries\slshape G}} $2$-complex. 
\begin{Verbatim}[commandchars=!@|,fontsize=\small,frame=single,label=Example]
  !gapprompt@gap>| !gapinput@K:=NewEquivariantTwoComplex(AlternatingGroup(5));|
  [ Alt( [ 1 .. 5 ] ), [  ], [  ], [  ], [  ] ]
\end{Verbatim}
 }

 

\subsection{\textcolor{Chapter }{AddOrbitOfVertices}}
\logpage{[ 1, 4, 2 ]}\nobreak
\hyperdef{L}{X7BC6EB4B84D3C733}{}
{\noindent\textcolor{FuncColor}{$\triangleright$\ \ \texttt{AddOrbitOfVertices({\mdseries\slshape K, H, label})\index{AddOrbitOfVertices@\texttt{AddOrbitOfVertices}}
\label{AddOrbitOfVertices}
}\hfill{\scriptsize (function)}}\\


 Adds an orbit of vertices to \mbox{\texttt{\mdseries\slshape K}} of type \mbox{\texttt{\mdseries\slshape G/H}}. Returns the new vertex corresponding to the coset \mbox{\texttt{\mdseries\slshape 1.H}}. A label \mbox{\texttt{\mdseries\slshape label}} for the new orbit must be provided. 
\begin{Verbatim}[commandchars=!@|,fontsize=\small,frame=single,label=Example]
  !gapprompt@gap>| !gapinput@v0 := AddOrbitOfVertices(K, Group((1,2)(3,4)), "A");|
  [ (), Group([ (1,2)(3,4) ]), "A" ]
\end{Verbatim}
 }

 

\subsection{\textcolor{Chapter }{AddOrbitOfEdges}}
\logpage{[ 1, 4, 3 ]}\nobreak
\hyperdef{L}{X7BA9A0257EF71486}{}
{\noindent\textcolor{FuncColor}{$\triangleright$\ \ \texttt{AddOrbitOfEdges({\mdseries\slshape K, H, v1, v2, label})\index{AddOrbitOfEdges@\texttt{AddOrbitOfEdges}}
\label{AddOrbitOfEdges}
}\hfill{\scriptsize (function)}}\\


 Adds an orbit of edges to \mbox{\texttt{\mdseries\slshape K}} of type \mbox{\texttt{\mdseries\slshape G/H}}. The vertices \mbox{\texttt{\mdseries\slshape v1,v2}} are the endpoints of the attached edge. Returns the new edge corresponding to
the coset \mbox{\texttt{\mdseries\slshape 1.H}}. A label \mbox{\texttt{\mdseries\slshape label}} for the new orbit must be provided. 
\begin{Verbatim}[commandchars=!@|,fontsize=\small,frame=single,label=Example]
  !gapprompt@gap>| !gapinput@e0 := AddOrbitOfEdges(K,Group(()),v0,v0,"some edges");|
  [ (), Group(()), "some edges", [ (), Group([ (1,2)(3,4) ]), "A" ], 
    [ (), Group([ (1,2)(3,4) ]), "A" ] ]
\end{Verbatim}
 }

 

\subsection{\textcolor{Chapter }{AddOrbitOfTwoCells}}
\logpage{[ 1, 4, 4 ]}\nobreak
\hyperdef{L}{X7F3CA2737AF8EA39}{}
{\noindent\textcolor{FuncColor}{$\triangleright$\ \ \texttt{AddOrbitOfTwoCells({\mdseries\slshape K, H, f, label})\index{AddOrbitOfTwoCells@\texttt{AddOrbitOfTwoCells}}
\label{AddOrbitOfTwoCells}
}\hfill{\scriptsize (function)}}\\


 Adds an orbit of $2$-cells to \mbox{\texttt{\mdseries\slshape K}} with stabilizer \mbox{\texttt{\mdseries\slshape H}} and attaching map \mbox{\texttt{\mdseries\slshape f}}. Returns the new 2-cell corresponding to the coset \mbox{\texttt{\mdseries\slshape 1.H}}. A label \mbox{\texttt{\mdseries\slshape label}} for the new orbit must be provided. 
\begin{Verbatim}[commandchars=!@|,fontsize=\small,frame=single,label=Example]
  !gapprompt@gap>| !gapinput@f:=[[e0,1],[e0,1],[e0,-1]];;|
  !gapprompt@gap>| !gapinput@AddOrbitOfTwoCells(K,Group(()),f,"some faces");|
  [ (), Group(()), "some faces", 
    [ 
        [ [ (), Group(()), "some edges", [ (), Group([ (1,2)(3,4) ]), "A" ], 
                [ (), Group([ (1,2)(3,4) ]), "A" ] ], 1 ], 
        [ [ (), Group(()), "some edges", [ (), Group([ (1,2)(3,4) ]), "A" ], 
                [ (), Group([ (1,2)(3,4) ]), "A" ] ], 1 ], 
        [ [ (), Group(()), "some edges", [ (), Group([ (1,2)(3,4) ]), "A" ], 
                [ (), Group([ (1,2)(3,4) ]), "A" ] ], -1 ] ] ]
\end{Verbatim}
 }

 }

  
\section{\textcolor{Chapter }{Action on the cells}}\logpage{[ 1, 5, 0 ]}
\hyperdef{L}{X794D2F797FDF40A0}{}
{
  

\subsection{\textcolor{Chapter }{ActionVertex}}
\logpage{[ 1, 5, 1 ]}\nobreak
\hyperdef{L}{X7AEB137A82CCD7FC}{}
{\noindent\textcolor{FuncColor}{$\triangleright$\ \ \texttt{ActionVertex({\mdseries\slshape g, v})\index{ActionVertex@\texttt{ActionVertex}}
\label{ActionVertex}
}\hfill{\scriptsize (function)}}\\


 Returns the vertex \mbox{\texttt{\mdseries\slshape g.v}}. }

 

\subsection{\textcolor{Chapter }{ActionEdge}}
\logpage{[ 1, 5, 2 ]}\nobreak
\hyperdef{L}{X7F30D1157DFDC5CB}{}
{\noindent\textcolor{FuncColor}{$\triangleright$\ \ \texttt{ActionEdge({\mdseries\slshape g, e})\index{ActionEdge@\texttt{ActionEdge}}
\label{ActionEdge}
}\hfill{\scriptsize (function)}}\\


 Returns the edge \mbox{\texttt{\mdseries\slshape g.e}}. }

 

\subsection{\textcolor{Chapter }{ActionOrientedEdge}}
\logpage{[ 1, 5, 3 ]}\nobreak
\hyperdef{L}{X7B97792D789B14D1}{}
{\noindent\textcolor{FuncColor}{$\triangleright$\ \ \texttt{ActionOrientedEdge({\mdseries\slshape g, e})\index{ActionOrientedEdge@\texttt{ActionOrientedEdge}}
\label{ActionOrientedEdge}
}\hfill{\scriptsize (function)}}\\


 Returns the oriented edge \mbox{\texttt{\mdseries\slshape g.e}}. }

 

\subsection{\textcolor{Chapter }{ActionEdgePath}}
\logpage{[ 1, 5, 4 ]}\nobreak
\hyperdef{L}{X8025EAF77D33CB17}{}
{\noindent\textcolor{FuncColor}{$\triangleright$\ \ \texttt{ActionEdgePath({\mdseries\slshape g, c})\index{ActionEdgePath@\texttt{ActionEdgePath}}
\label{ActionEdgePath}
}\hfill{\scriptsize (function)}}\\


 Returns the edge path \mbox{\texttt{\mdseries\slshape g.c}}. }

 

\subsection{\textcolor{Chapter }{ActionTwoCell}}
\logpage{[ 1, 5, 5 ]}\nobreak
\hyperdef{L}{X7DEE307785F5C43B}{}
{\noindent\textcolor{FuncColor}{$\triangleright$\ \ \texttt{ActionTwoCell({\mdseries\slshape g, f})\index{ActionTwoCell@\texttt{ActionTwoCell}}
\label{ActionTwoCell}
}\hfill{\scriptsize (function)}}\\


 Returns the 2-cell \mbox{\texttt{\mdseries\slshape g.f}}. }

 }

  
\section{\textcolor{Chapter }{Stabilizers}}\logpage{[ 1, 6, 0 ]}
\hyperdef{L}{X797BD60E7ACEF1B1}{}
{
  

\subsection{\textcolor{Chapter }{StabilizerVertex}}
\logpage{[ 1, 6, 1 ]}\nobreak
\hyperdef{L}{X87F41FE184271A87}{}
{\noindent\textcolor{FuncColor}{$\triangleright$\ \ \texttt{StabilizerVertex({\mdseries\slshape v})\index{StabilizerVertex@\texttt{StabilizerVertex}}
\label{StabilizerVertex}
}\hfill{\scriptsize (function)}}\\


 Returns the stabilizer of \mbox{\texttt{\mdseries\slshape v}}. }

 

\subsection{\textcolor{Chapter }{StabilizerEdge}}
\logpage{[ 1, 6, 2 ]}\nobreak
\hyperdef{L}{X85990F2779642A83}{}
{\noindent\textcolor{FuncColor}{$\triangleright$\ \ \texttt{StabilizerEdge({\mdseries\slshape e})\index{StabilizerEdge@\texttt{StabilizerEdge}}
\label{StabilizerEdge}
}\hfill{\scriptsize (function)}}\\


 Returns the stabilizer of \mbox{\texttt{\mdseries\slshape e}}. }

 

\subsection{\textcolor{Chapter }{StabilizerOrientedEdge}}
\logpage{[ 1, 6, 3 ]}\nobreak
\hyperdef{L}{X7A1C6C0C7CCF5890}{}
{\noindent\textcolor{FuncColor}{$\triangleright$\ \ \texttt{StabilizerOrientedEdge({\mdseries\slshape e})\index{StabilizerOrientedEdge@\texttt{StabilizerOrientedEdge}}
\label{StabilizerOrientedEdge}
}\hfill{\scriptsize (function)}}\\


 Returns the stabilizer of \mbox{\texttt{\mdseries\slshape e}}. }

 

\subsection{\textcolor{Chapter }{StabilizerEdgePath}}
\logpage{[ 1, 6, 4 ]}\nobreak
\hyperdef{L}{X7CCFA20278B2A82B}{}
{\noindent\textcolor{FuncColor}{$\triangleright$\ \ \texttt{StabilizerEdgePath({\mdseries\slshape g, f})\index{StabilizerEdgePath@\texttt{StabilizerEdgePath}}
\label{StabilizerEdgePath}
}\hfill{\scriptsize (function)}}\\


 Returns the 2-cell \mbox{\texttt{\mdseries\slshape g.f}}. }

 

\subsection{\textcolor{Chapter }{StabilizerTwoCell}}
\logpage{[ 1, 6, 5 ]}\nobreak
\hyperdef{L}{X81B88CC0860DDA72}{}
{\noindent\textcolor{FuncColor}{$\triangleright$\ \ \texttt{StabilizerTwoCell({\mdseries\slshape f})\index{StabilizerTwoCell@\texttt{StabilizerTwoCell}}
\label{StabilizerTwoCell}
}\hfill{\scriptsize (function)}}\\


 Returns the stabilizer of \mbox{\texttt{\mdseries\slshape f}}. }

 

\subsection{\textcolor{Chapter }{StabilizerCell}}
\logpage{[ 1, 6, 6 ]}\nobreak
\hyperdef{L}{X7E711A6F7ACCD6EF}{}
{\noindent\textcolor{FuncColor}{$\triangleright$\ \ \texttt{StabilizerCell({\mdseries\slshape e})\index{StabilizerCell@\texttt{StabilizerCell}}
\label{StabilizerCell}
}\hfill{\scriptsize (function)}}\\


 Returns the stabilizer of a $k$-cell \mbox{\texttt{\mdseries\slshape e}}. }

 }

  
\section{\textcolor{Chapter }{Edges and edge paths}}\logpage{[ 1, 7, 0 ]}
\hyperdef{L}{X864DA5D685A65A0C}{}
{
  

\subsection{\textcolor{Chapter }{VerticesOfEdge}}
\logpage{[ 1, 7, 1 ]}\nobreak
\hyperdef{L}{X7A0125FA84C31024}{}
{\noindent\textcolor{FuncColor}{$\triangleright$\ \ \texttt{VerticesOfEdge({\mdseries\slshape e})\index{VerticesOfEdge@\texttt{VerticesOfEdge}}
\label{VerticesOfEdge}
}\hfill{\scriptsize (function)}}\\


 Returns the set of vertices of the edge \mbox{\texttt{\mdseries\slshape e}}. }

 

\subsection{\textcolor{Chapter }{SourceOrientedEdge}}
\logpage{[ 1, 7, 2 ]}\nobreak
\hyperdef{L}{X7FEDA4BB7FF50D43}{}
{\noindent\textcolor{FuncColor}{$\triangleright$\ \ \texttt{SourceOrientedEdge({\mdseries\slshape e})\index{SourceOrientedEdge@\texttt{SourceOrientedEdge}}
\label{SourceOrientedEdge}
}\hfill{\scriptsize (function)}}\\


 Returns the source of the oriented edge \mbox{\texttt{\mdseries\slshape e}}. }

 

\subsection{\textcolor{Chapter }{TargetOrientedEdge}}
\logpage{[ 1, 7, 3 ]}\nobreak
\hyperdef{L}{X8459F5417DAF5F9A}{}
{\noindent\textcolor{FuncColor}{$\triangleright$\ \ \texttt{TargetOrientedEdge({\mdseries\slshape e})\index{TargetOrientedEdge@\texttt{TargetOrientedEdge}}
\label{TargetOrientedEdge}
}\hfill{\scriptsize (function)}}\\


 Returns the target of the oriented edge \mbox{\texttt{\mdseries\slshape e}}. }

 

\subsection{\textcolor{Chapter }{VerticesOrientedEdge}}
\logpage{[ 1, 7, 4 ]}\nobreak
\hyperdef{L}{X82D700747E67DDBC}{}
{\noindent\textcolor{FuncColor}{$\triangleright$\ \ \texttt{VerticesOrientedEdge({\mdseries\slshape e})\index{VerticesOrientedEdge@\texttt{VerticesOrientedEdge}}
\label{VerticesOrientedEdge}
}\hfill{\scriptsize (function)}}\\


 Returns a list with the source and target of the oriented edge \mbox{\texttt{\mdseries\slshape e}} (in this order). }

 

\subsection{\textcolor{Chapter }{OppositeEdge}}
\logpage{[ 1, 7, 5 ]}\nobreak
\hyperdef{L}{X829C6E66782A644C}{}
{\noindent\textcolor{FuncColor}{$\triangleright$\ \ \texttt{OppositeEdge({\mdseries\slshape e})\index{OppositeEdge@\texttt{OppositeEdge}}
\label{OppositeEdge}
}\hfill{\scriptsize (function)}}\\


 Returns the opposite edge of an oriented edge \mbox{\texttt{\mdseries\slshape e}}. }

 

\subsection{\textcolor{Chapter }{IsEdgePath}}
\logpage{[ 1, 7, 6 ]}\nobreak
\hyperdef{L}{X8780DA8B83349244}{}
{\noindent\textcolor{FuncColor}{$\triangleright$\ \ \texttt{IsEdgePath({\mdseries\slshape c})\index{IsEdgePath@\texttt{IsEdgePath}}
\label{IsEdgePath}
}\hfill{\scriptsize (function)}}\\


 Checks if a list of edges \mbox{\texttt{\mdseries\slshape c}} is an edge path. }

 

\subsection{\textcolor{Chapter }{IsClosedEdgePath}}
\logpage{[ 1, 7, 7 ]}\nobreak
\hyperdef{L}{X8735FB3881C631FC}{}
{\noindent\textcolor{FuncColor}{$\triangleright$\ \ \texttt{IsClosedEdgePath({\mdseries\slshape c})\index{IsClosedEdgePath@\texttt{IsClosedEdgePath}}
\label{IsClosedEdgePath}
}\hfill{\scriptsize (function)}}\\


 Checks if a list of edges \mbox{\texttt{\mdseries\slshape c}} is a closed edge path. }

 

\subsection{\textcolor{Chapter }{InverseEdgePath}}
\logpage{[ 1, 7, 8 ]}\nobreak
\hyperdef{L}{X7CC10161877262D4}{}
{\noindent\textcolor{FuncColor}{$\triangleright$\ \ \texttt{InverseEdgePath({\mdseries\slshape c})\index{InverseEdgePath@\texttt{InverseEdgePath}}
\label{InverseEdgePath}
}\hfill{\scriptsize (function)}}\\


 Returns the inverse edge path of an edge path \mbox{\texttt{\mdseries\slshape c}}. }

 

\subsection{\textcolor{Chapter }{ReducedEdgePath}}
\logpage{[ 1, 7, 9 ]}\nobreak
\hyperdef{L}{X7934E14C87F7449E}{}
{\noindent\textcolor{FuncColor}{$\triangleright$\ \ \texttt{ReducedEdgePath({\mdseries\slshape c})\index{ReducedEdgePath@\texttt{ReducedEdgePath}}
\label{ReducedEdgePath}
}\hfill{\scriptsize (function)}}\\


 Reduces the edge path \mbox{\texttt{\mdseries\slshape c}} (destructive). }

 

\subsection{\textcolor{Chapter }{CyclicallyReducedEdgePath}}
\logpage{[ 1, 7, 10 ]}\nobreak
\hyperdef{L}{X8560677879A17255}{}
{\noindent\textcolor{FuncColor}{$\triangleright$\ \ \texttt{CyclicallyReducedEdgePath({\mdseries\slshape c})\index{CyclicallyReducedEdgePath@\texttt{CyclicallyReducedEdgePath}}
\label{CyclicallyReducedEdgePath}
}\hfill{\scriptsize (function)}}\\


 Cyclically reduces the edge path \mbox{\texttt{\mdseries\slshape c}} (destructive). }

 }

  
\section{\textcolor{Chapter }{Complex mod G}}\logpage{[ 1, 8, 0 ]}
\hyperdef{L}{X7FDB38217B1C1A5D}{}
{
  These functions allow to work with the complex $K/G$. 

\subsection{\textcolor{Chapter }{TwoComplexModG}}
\logpage{[ 1, 8, 1 ]}\nobreak
\hyperdef{L}{X792C532181D709E6}{}
{\noindent\textcolor{FuncColor}{$\triangleright$\ \ \texttt{TwoComplexModG({\mdseries\slshape K})\index{TwoComplexModG@\texttt{TwoComplexModG}}
\label{TwoComplexModG}
}\hfill{\scriptsize (function)}}\\


 Returns the complex \mbox{\texttt{\mdseries\slshape K/G}}. This is represented as a 2-complex with an action of the trivial group. 
\begin{Verbatim}[commandchars=!@|,fontsize=\small,frame=single,label=Example]
  !gapprompt@gap>| !gapinput@KmodG:=TwoComplexModG(K);|
\end{Verbatim}
 }

 

\subsection{\textcolor{Chapter }{VertexModG}}
\logpage{[ 1, 8, 2 ]}\nobreak
\hyperdef{L}{X7C7FAB2686069CAE}{}
{\noindent\textcolor{FuncColor}{$\triangleright$\ \ \texttt{VertexModG({\mdseries\slshape v})\index{VertexModG@\texttt{VertexModG}}
\label{VertexModG}
}\hfill{\scriptsize (function)}}\\


 }

 

\subsection{\textcolor{Chapter }{EdgeModG}}
\logpage{[ 1, 8, 3 ]}\nobreak
\hyperdef{L}{X8412A2F97FA85F93}{}
{\noindent\textcolor{FuncColor}{$\triangleright$\ \ \texttt{EdgeModG({\mdseries\slshape e})\index{EdgeModG@\texttt{EdgeModG}}
\label{EdgeModG}
}\hfill{\scriptsize (function)}}\\


 }

 

\subsection{\textcolor{Chapter }{DirectedEdgeModG}}
\logpage{[ 1, 8, 4 ]}\nobreak
\hyperdef{L}{X87BF131F798DEA98}{}
{\noindent\textcolor{FuncColor}{$\triangleright$\ \ \texttt{DirectedEdgeModG({\mdseries\slshape e})\index{DirectedEdgeModG@\texttt{DirectedEdgeModG}}
\label{DirectedEdgeModG}
}\hfill{\scriptsize (function)}}\\


 }

 

\subsection{\textcolor{Chapter }{EdgePathModG}}
\logpage{[ 1, 8, 5 ]}\nobreak
\hyperdef{L}{X7C4B0D1983C1E630}{}
{\noindent\textcolor{FuncColor}{$\triangleright$\ \ \texttt{EdgePathModG({\mdseries\slshape c})\index{EdgePathModG@\texttt{EdgePathModG}}
\label{EdgePathModG}
}\hfill{\scriptsize (function)}}\\


 }

 }

  
\section{\textcolor{Chapter }{Fundamental group}}\logpage{[ 1, 9, 0 ]}
\hyperdef{L}{X849DA0DB7AA01EF7}{}
{
  

\subsection{\textcolor{Chapter }{Pi1}}
\logpage{[ 1, 9, 1 ]}\nobreak
\hyperdef{L}{X826A78F18562419E}{}
{\noindent\textcolor{FuncColor}{$\triangleright$\ \ \texttt{Pi1({\mdseries\slshape K[, T]})\index{Pi1@\texttt{Pi1}}
\label{Pi1}
}\hfill{\scriptsize (function)}}\\


 Returns the fundamental group of a connected complex \mbox{\texttt{\mdseries\slshape K}}. Optionally, to compute the fundamental group a specific spanning tree \mbox{\texttt{\mdseries\slshape T}} may be provided. Returns \mbox{\texttt{\mdseries\slshape fail}} if \mbox{\texttt{\mdseries\slshape K}} is not connected. }

 

\subsection{\textcolor{Chapter }{ElementOfPi1FromClosedEdgePath}}
\logpage{[ 1, 9, 2 ]}\nobreak
\hyperdef{L}{X874C512F8200A75C}{}
{\noindent\textcolor{FuncColor}{$\triangleright$\ \ \texttt{ElementOfPi1FromClosedEdgePath({\mdseries\slshape K, c})\index{ElementOfPi1FromClosedEdgePath@\texttt{ElementOfPi1FromClosedEdgePath}}
\label{ElementOfPi1FromClosedEdgePath}
}\hfill{\scriptsize (function)}}\\


 Returns the element of the fundamental group of \mbox{\texttt{\mdseries\slshape K}} representing the class of the closed edge path \mbox{\texttt{\mdseries\slshape c}}. }

 

\subsection{\textcolor{Chapter }{Pi1RandomSpanningTree}}
\logpage{[ 1, 9, 3 ]}\nobreak
\hyperdef{L}{X79FE94A085FD231B}{}
{\noindent\textcolor{FuncColor}{$\triangleright$\ \ \texttt{Pi1RandomSpanningTree({\mdseries\slshape K})\index{Pi1RandomSpanningTree@\texttt{Pi1RandomSpanningTree}}
\label{Pi1RandomSpanningTree}
}\hfill{\scriptsize (function)}}\\


 Returns the fundamental group of a connected complex \mbox{\texttt{\mdseries\slshape K}}, computed using a random spanning tree of \mbox{\texttt{\mdseries\slshape K}}. Returns \mbox{\texttt{\mdseries\slshape fail}} if \mbox{\texttt{\mdseries\slshape K}} is not connected. }

 

\subsection{\textcolor{Chapter }{Pi1XModX0}}
\logpage{[ 1, 9, 4 ]}\nobreak
\hyperdef{L}{X7906414581365332}{}
{\noindent\textcolor{FuncColor}{$\triangleright$\ \ \texttt{Pi1XModX0({\mdseries\slshape K})\index{Pi1XModX0@\texttt{Pi1XModX0}}
\label{Pi1XModX0}
}\hfill{\scriptsize (function)}}\\


 Returns the fundamental group of \mbox{\texttt{\mdseries\slshape K/K\texttt{\symbol{94}}0}}. }

 

\subsection{\textcolor{Chapter }{SpanningTreeOfComplex}}
\logpage{[ 1, 9, 5 ]}\nobreak
\hyperdef{L}{X79A1312678A48B2B}{}
{\noindent\textcolor{FuncColor}{$\triangleright$\ \ \texttt{SpanningTreeOfComplex({\mdseries\slshape K})\index{SpanningTreeOfComplex@\texttt{SpanningTreeOfComplex}}
\label{SpanningTreeOfComplex}
}\hfill{\scriptsize (function)}}\\


 Returns a spanning tree for the 1-skeleton of \mbox{\texttt{\mdseries\slshape K}}. }

 

\subsection{\textcolor{Chapter }{RandomSpanningTreeOfComplex}}
\logpage{[ 1, 9, 6 ]}\nobreak
\hyperdef{L}{X85A097E8814EF371}{}
{\noindent\textcolor{FuncColor}{$\triangleright$\ \ \texttt{RandomSpanningTreeOfComplex({\mdseries\slshape K})\index{RandomSpanningTreeOfComplex@\texttt{RandomSpanningTreeOfComplex}}
\label{RandomSpanningTreeOfComplex}
}\hfill{\scriptsize (function)}}\\


 Returns a spanning tree for the 1-skeleton of \mbox{\texttt{\mdseries\slshape K}} chosen randomly. }

 

\subsection{\textcolor{Chapter }{IsSpanningTreeOfComplex}}
\logpage{[ 1, 9, 7 ]}\nobreak
\hyperdef{L}{X78C2267E7C3E0671}{}
{\noindent\textcolor{FuncColor}{$\triangleright$\ \ \texttt{IsSpanningTreeOfComplex({\mdseries\slshape K, T})\index{IsSpanningTreeOfComplex@\texttt{IsSpanningTreeOfComplex}}
\label{IsSpanningTreeOfComplex}
}\hfill{\scriptsize (function)}}\\


 Returns \texttt{true} if \mbox{\texttt{\mdseries\slshape T}} is a spanning tree of \mbox{\texttt{\mdseries\slshape K}}, \texttt{false} otherwise. }

 }

  
\section{\textcolor{Chapter }{Homotopy properties}}\logpage{[ 1, 10, 0 ]}
\hyperdef{L}{X7D282DB47BD4C702}{}
{
  

\subsection{\textcolor{Chapter }{IsAcyclic}}
\logpage{[ 1, 10, 1 ]}\nobreak
\hyperdef{L}{X847A62A6806046C4}{}
{\noindent\textcolor{FuncColor}{$\triangleright$\ \ \texttt{IsAcyclic({\mdseries\slshape K})\index{IsAcyclic@\texttt{IsAcyclic}}
\label{IsAcyclic}
}\hfill{\scriptsize (function)}}\\


 Returns \texttt{true} if \mbox{\texttt{\mdseries\slshape K}} is acyclic, \texttt{false} otherwise. }

 

\subsection{\textcolor{Chapter }{IsAsphericalComplex}}
\logpage{[ 1, 10, 2 ]}\nobreak
\hyperdef{L}{X7A514C8E800BABD1}{}
{\noindent\textcolor{FuncColor}{$\triangleright$\ \ \texttt{IsAsphericalComplex({\mdseries\slshape K})\index{IsAsphericalComplex@\texttt{IsAsphericalComplex}}
\label{IsAsphericalComplex}
}\hfill{\scriptsize (function)}}\\


 If it returns \texttt{true}, then \mbox{\texttt{\mdseries\slshape K}} is aspherical. It may return \texttt{fail}. Uses the function \texttt{IsAspherical} from the package \textsf{HAP}. }

 

\subsection{\textcolor{Chapter }{IsContractible}}
\logpage{[ 1, 10, 3 ]}\nobreak
\hyperdef{L}{X7F2B3D367B0C8F3D}{}
{\noindent\textcolor{FuncColor}{$\triangleright$\ \ \texttt{IsContractible({\mdseries\slshape K[, time{\textunderscore}limit]})\index{IsContractible@\texttt{IsContractible}}
\label{IsContractible}
}\hfill{\scriptsize (function)}}\\


 If it returns \texttt{true}, then \mbox{\texttt{\mdseries\slshape K}} is contractible. It may return \texttt{fail}. The optional argument \mbox{\texttt{\mdseries\slshape time{\textunderscore}limit}} allows to set a time limit for the computation of the fundamental group. }

 }

  
\section{\textcolor{Chapter }{Random attaching maps}}\logpage{[ 1, 11, 0 ]}
\hyperdef{L}{X7953F783878A6A62}{}
{
  

\subsection{\textcolor{Chapter }{RandomAttachingMaps}}
\logpage{[ 1, 11, 1 ]}\nobreak
\hyperdef{L}{X7B53C9FE7EBD6E4A}{}
{\noindent\textcolor{FuncColor}{$\triangleright$\ \ \texttt{RandomAttachingMaps({\mdseries\slshape K, lengths})\index{RandomAttachingMaps@\texttt{RandomAttachingMaps}}
\label{RandomAttachingMaps}
}\hfill{\scriptsize (function)}}\\


 Returns a list of randomly chosen closed edge paths in \mbox{\texttt{\mdseries\slshape K}} of the specified lengths \mbox{\texttt{\mdseries\slshape lenghts}}. }

 }

  
\section{\textcolor{Chapter }{More}}\logpage{[ 1, 12, 0 ]}
\hyperdef{L}{X8243F75A804C2894}{}
{
  

\subsection{\textcolor{Chapter }{H2AsGModule}}
\logpage{[ 1, 12, 1 ]}\nobreak
\hyperdef{L}{X7A1893137C5658CC}{}
{\noindent\textcolor{FuncColor}{$\triangleright$\ \ \texttt{H2AsGModule({\mdseries\slshape K})\index{H2AsGModule@\texttt{H2AsGModule}}
\label{H2AsGModule}
}\hfill{\scriptsize (function)}}\\


 Returns the representation of the group \mbox{\texttt{\mdseries\slshape G}} given by the action on \mbox{\texttt{\mdseries\slshape H{\textunderscore}2(K)}}. It is represented as a morphism \mbox{\texttt{\mdseries\slshape G -{\textgreater} GL(m,Z)}} where \mbox{\texttt{\mdseries\slshape m}} is the rank of \mbox{\texttt{\mdseries\slshape H{\textunderscore}2(K)}}. }

 

\subsection{\textcolor{Chapter }{CoveringSpaceFromHomomorphism}}
\logpage{[ 1, 12, 2 ]}\nobreak
\hyperdef{L}{X7DF3E71C7B84C323}{}
{\noindent\textcolor{FuncColor}{$\triangleright$\ \ \texttt{CoveringSpaceFromHomomorphism({\mdseries\slshape H, G, phi})\index{CoveringSpaceFromHomomorphism@\texttt{CoveringSpaceFromHomomorphism}}
\label{CoveringSpaceFromHomomorphism}
}\hfill{\scriptsize (function)}}\\


 If \mbox{\texttt{\mdseries\slshape H}} is finitely presented, \mbox{\texttt{\mdseries\slshape G}} is finite and \mbox{\texttt{\mdseries\slshape phi}} is an epimorphism \mbox{\texttt{\mdseries\slshape f: H -{\textgreater} G}}, returns the covering space of the presentation complex of \mbox{\texttt{\mdseries\slshape H}} corresponding to the subgroup \texttt{Kernel(phi)} of \mbox{\texttt{\mdseries\slshape H}}. This covering space is represented as a \mbox{\texttt{\mdseries\slshape G}} 2-complex. }

 }

  
\section{\textcolor{Chapter }{Additional functions}}\logpage{[ 1, 13, 0 ]}
\hyperdef{L}{X80E9EA5E866890E8}{}
{
  

\subsection{\textcolor{Chapter }{CanonicalLeftCosetElement}}
\logpage{[ 1, 13, 1 ]}\nobreak
\hyperdef{L}{X7C21BFA77F5FA6D9}{}
{\noindent\textcolor{FuncColor}{$\triangleright$\ \ \texttt{CanonicalLeftCosetElement({\mdseries\slshape g, H})\index{CanonicalLeftCosetElement@\texttt{CanonicalLeftCosetElement}}
\label{CanonicalLeftCosetElement}
}\hfill{\scriptsize (function)}}\\


 Returns a "canonical" representative of the left coset \mbox{\texttt{\mdseries\slshape gH}} which is independent of the given representative \mbox{\texttt{\mdseries\slshape g}}. This can be used to compare cosets by comparing their canonical
representatives. The representative chosen to be the "canonical" one is
representation dependent and only guaranteed to remain the same within one \textsf{GAP} session. See also \texttt{CanonicalRightCosetElement} (\textbf{Reference: CanonicalRightCosetElement}) 
\begin{Verbatim}[commandchars=!@|,fontsize=\small,frame=single,label=Example]
  !gapprompt@gap>| !gapinput@CanonicalLeftCosetElement((2,3,5),H);|
  (3,5)
\end{Verbatim}
 }

 

\subsection{\textcolor{Chapter }{Epimorphism}}
\logpage{[ 1, 13, 2 ]}\nobreak
\hyperdef{L}{X79D0FB7D7E515244}{}
{\noindent\textcolor{FuncColor}{$\triangleright$\ \ \texttt{Epimorphism({\mdseries\slshape P, G})\index{Epimorphism@\texttt{Epimorphism}}
\label{Epimorphism}
}\hfill{\scriptsize (function)}}\\


 Returns \texttt{true} if there is an epimorphism from the group given by the presentation \mbox{\texttt{\mdseries\slshape P}} to the finite group \mbox{\texttt{\mdseries\slshape G}}. Otherwise returns \texttt{false}. }

 

\subsection{\textcolor{Chapter }{PresentationLength}}
\logpage{[ 1, 13, 3 ]}\nobreak
\hyperdef{L}{X7F0EC10B87511BD6}{}
{\noindent\textcolor{FuncColor}{$\triangleright$\ \ \texttt{PresentationLength({\mdseries\slshape P})\index{PresentationLength@\texttt{PresentationLength}}
\label{PresentationLength}
}\hfill{\scriptsize (function)}}\\


 Returns the length of the presentation \mbox{\texttt{\mdseries\slshape P}}. }

 }

  }

 \def\bibname{References\logpage{[ "Bib", 0, 0 ]}
\hyperdef{L}{X7A6F98FD85F02BFE}{}
}

\bibliographystyle{alpha}
\bibliography{g2comp}

\addcontentsline{toc}{chapter}{References}

\def\indexname{Index\logpage{[ "Ind", 0, 0 ]}
\hyperdef{L}{X83A0356F839C696F}{}
}

\cleardoublepage
\phantomsection
\addcontentsline{toc}{chapter}{Index}


\printindex

\newpage
\immediate\write\pagenrlog{["End"], \arabic{page}];}
\immediate\closeout\pagenrlog
\end{document}
